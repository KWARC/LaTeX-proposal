% the document class specification for the proposal writing process, add the 'submit' option
% for submitting (switches off various draft features); add the 'public' option to exclude
% any private parts. 
\documentclass[gitinfo]{dfgreporting}
%\documentclass[submit]{dfgproposal}
%\documentclass[submit,public]{dfgproposal}
\addbibresource{../lib/dummy.bib}
\renewcommand{\familydefault}{\sfdefault}
\usepackage[scaled=.90]{helvet}

\begin{document}

\begin{center}\color{red}\huge
  This mock report is just an example for \texttt{dfgreporting.cls} it reflects the
  template valid until January 2012.
\end{center}

\begin{Large}\color{red}
  The first time this proposal file is being run it will throw errors (forward references
  need to be initialized), just run pdflatex on it again.

  This report also depends on the file \texttt{../proposal/proposal.pdata}: it imports
  metadata from there to save you typing and keep things in sync (see the
  \texttt{importsfrom=} option). If that file is missing, re-generate that by running
  \texttt{pdflatex} (twice) on ../proposal/proposal.
\end{Large}

\begin{report}[
  importfrom=../proposal/proposal,
  key = KO 2428 99-9,
  key = GS 4711 99-9,
  thema=Intelligentes Schreiben von Antr\"agen, 
  reportperiod=1. Feb. 2010 - 31. Jan. 2012,
  instrument=Final Project Report, 
  % fauemployed=Junior Researcher: 1. Feb 2010 - 31. Jan 2012,
  % pcgemployed=Slave Worker: 1. Feb 2010 - 31. Dec 2010,
  % pcgemployed=Lazy Bones: 1.Jan 2011 - 31. Jan 2012,
  applareas={Knowledge Management, Document Management, Workflow Systems},
  % coop={Acquisition Guru, Berlin, Germany},
  % coop={Deutsche Forschungsgemeinschaft, Bonn, Germany},
  acronym={ABC},
  discipline={Elektrotechnik},
  areas={Ingenieurwissenschaften},
  projpapers={Kohlhase:pdpl10,providemore}]
\section{Final Progress Report}
\begin{todo}{from the report template}
This is what the reviewers read  (maximum 10 pages of A4)
  \begin{itemize}
  \item Project’s initial questions and objectives.
  \item Project developments --- including deviations from the original plan, failures,
    and problems encountered with project organisation or technical execution.
  \item Presentation of results and discussion of the relevant research situation in this
    context, potential perspectives for application, and conceivable follow-up research.
  \item Statement on whether the results of the project are economically valuable and whether exploitation is already taking place or may be anticipated; if applicable, details regarding patents, industrial joint ventures, etc.
  \item Who has contributed to the results achieved by the project (national/international partners, project staff, etc.)?
  \item Qualification of young researchers in the context of your project (for example, first degree, doctorate, post-doctorate, etc.).
  \end{itemize}
  The report must be understandable without the need to consult additional literature. To
  illustrate and enhance your presentation you may refer to your own and others’
  publications. Make it clear whenever you are referring to other researchers’ work and
  explain your own papers. Please list all cited publications at the end of the
  section. This reference list is not considered your list of publications. Any
  unpublished work must be included with the final report. However, note that reviewers
  are not required to read any of the works you cite. Reviews will be based only on the
  text of the actual report.
\end{todo}
%%% Local Variables: 
%%% mode: LaTeX
%%% TeX-master: "finalreport"
%%% End: 

% LocalWords:  finalreport progressreport.tex

\section{Final Progress Report}
\begin{todo}{from the report template}
  This is for the DFG web site and report, made available to the general public (maximum 1 page of A4)
  \begin{itemize}
  \item Presentation, in clearly understandable, everyday language of the key scientific
    findings and any potential applications.
  \item Any surprises encountered in the course of the project and in the results
    obtained.
  \item Reference to any articles published in the media reporting the success of the
    project.Project’s initial questions and objectives.
  \end{itemize}
\end{todo}
%%% Local Variables: 
%%% mode: latex
%%% TeX-master: "finalreport"
%%% End: 

% LocalWords:  finalreport

\dfgprojpapers[articles,confpapers,wspapers]{Kohlhase:pdpl10,providemore,KohDavGin:psewads11,Lange:OpenMathCDLinkedData10}
\end{report}
\end{document}
 
%%% Local Variables: 
%%% mode: LaTeX
%%% TeX-master: t
%%% End: 
