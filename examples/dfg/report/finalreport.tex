% the document class specification for the proposal writing process, add the 'submit' option
% for submitting (switches off various draft features); add the 'public' option to exclude
% any private parts. 
\documentclass[gitinfo]{dfgreporting}
%\documentclass[submit]{dfgproposal}
%\documentclass[submit,public]{dfgproposal}
\addbibresource{../lib/dummy.bib}
\renewcommand{\familydefault}{\sfdefault}
\usepackage[scaled=.90]{helvet}

\begin{document}

\begin{center}\color{red}\huge
  This mock report is just an example for \texttt{dfgreporting.cls} it reflects the
  template valid until January 2012.
\end{center}

\begin{Large}\color{red}
  The first time this proposal file is being run it will throw errors (forward references
  need to be initialized), just run pdflatex on it again.

  This report also depends on the file \texttt{../proposal/proposal.pdata}: it imports
  metadata from there to save you typing and keep things in sync (see the
  \texttt{importsfrom=} option). If that file is missing, re-generate that by running
  \texttt{pdflatex} (twice) on ../proposal/proposal.
\end{Large}

\begin{report}[
  importfrom=../proposal/proposal,
  key = KO 2428 99-9,
  key = GS 4711 99-9,
  thema=Intelligentes Schreiben von Antr\"agen, 
  reportperiod=1. Feb. 2010 - 31. Jan. 2012,
  instrument=Final Project Report, 
  % fauemployed=Junior Researcher: 1. Feb 2010 - 31. Jan 2012,
  % pcgemployed=Slave Worker: 1. Feb 2010 - 31. Dec 2010,
  % pcgemployed=Lazy Bones: 1.Jan 2011 - 31. Jan 2012,
  applareas={Knowledge Management, Document Management, Workflow Systems},
  % coop={Acquisition Guru, Berlin, Germany},
  % coop={Deutsche Forschungsgemeinschaft, Bonn, Germany},
  acronym={ABC},
  discipline={Elektrotechnik},
  areas={Ingenieurwissenschaften},
  projpapers={Kohlhase:pdpl10,providemore}]
\include{progressreport}
\include{progresssummary}
\dfgprojpapers[articles,confpapers,wspapers]{Kohlhase:pdpl10,providemore,KohDavGin:psewads11,Lange:OpenMathCDLinkedData10}
\end{report}
\end{document}
 
%%% Local Variables: 
%%% mode: LaTeX
%%% TeX-master: t
%%% End: 
