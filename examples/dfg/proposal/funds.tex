\section{People/collaborations/funding}\label{sec:funds}

\subsection{Employment status information \deu{(Angaben zur Dienststellung)}}
\begin{todo}{from the proposal template}
For each applicant, state the last name, first name, and employment status (including
duration of contract and funding body, if on a fixed-term contract).
\end{todo}

\subsection{First-time proposal data \deu{(Angaben zur Erstantragstellung)}}
\begin{todo}{from the proposal template}
Only if applicable: Last name, first name of first-time applicant.

If this is your first proposal, reviewers will consider this fact when assessing your pro-
posal. Previous proposals for research fellowships, publication funding, travel allow-
ances, or funding for scientific networks are not considered first proposals. If you are
submitting a “first-time proposal” and it is part of a joint proposal, please note that your
independent project must be distinct from the other projects.

If you have already submitted a proposal as an applicant for a research grant and have
received a letter informing you of the funding decision, or if you have led an independ-
ent junior research group or project in a Collaborative Research Centre or Research
Unit, you are no longer eligible to submit a “first proposal”. If you have submitted a
“first-time proposal” and it was rejected, you may resubmit the application, in revised
form, as a first-time proposal for the same project.
\end{todo}

\subsection{Composition of the project group \deu{(Zusammensetzung der Projektarbeitsgruppe)}}
\begin{todo}{from the proposal template}
List only those individuals who will work on the project but will not be paid out of the
project funds. State each person’s name, academic title, employment status, and type
of funding.

Please list separately the individuals paid by your institution and those paid using other
third-party funding (including fellowships).
\end{todo}
The KWARC (Knowledge Adaptation and Reasoning for Content) research group headed by
Michael Kohlhase for has the following members
\begin{compactdesc}
\item[Dr. N.N.] is the \ldots She has a background in\ldots.
\end{compactdesc}
Additionally, the group has attracted about 10 undergraduate and master's students that
actively take part in the project work and various aspects of research.

\subsection{Researchers in Germany with whom you have agreed to cooperate on this project}

\begin{compactdesc}
\item[Prof. Dr. Super Akquisiteur (Uni Paderborn)] knows exactly what to do to get funding
  with DFG, we will interview him closely and integrate all his intuitions into the
  {\pn} templates.
\item[Prof. Dr. Habe Nichts (Uni Hinterpfuiteufel)] has never gotten a grant proposal
  through with DFG, we will try to avoid his mistakes.
\item[Dr. Sach Bearbeiter (DFG)] will consult with the DFG requirements to be met in the
  proposals.
\item[Dr. Donald Knuth (Stanford University)] is so surprised that we want to do grant
  proposals in {\TeX/\LaTeX} that he will help us with any problems we have in coding in
  this wonderful programming language.
\end{compactdesc}

\subsection{Researchers abroad with whom you have agreed to cooperate on this project}

\begin{compactdesc}
\item[Dr. Donald Knuth (Stanford University)] is so surprised that we want to do grant
  proposals in {\TeX/\LaTeX} that he will help us with any problems we have in coding in
  this wonderful programming language.
\end{compactdesc}

\subsubsection{Researchers with whom you have collaborated scientifically within the past three years \deu{(Wissenschaftlerinnen und Wissenschaftler, mit denen in den letzten drei Jahren wissenschaftlich zusammengearbeitet wurde)}}

\ednote{Anmerkung Jens: Etwas unklar, was die DFG hier möchte. Die Liste der Personen kann
  sehr lang sein, also ist es wahrscheinlich besser nur die wichtigsten Projekte und
  Kontakte zu listen.}

\begin{todo}{from the proposal template}
This information will assist the DFG’s Head Office in avoiding potential conflicts of interest during the review process.
\end{todo}

\subsection{Project-relevant cooperation with commercial enterprises}
Not applicable.

\subsection{Project-relevant participation in commercial enterprises}
Not applicable.

\subsection{Scientific equipment \deu{(Apparative Ausstattung)}}

Jacobs University provides laptops or desktop workstations for all academic
employees. Great Consulting GmbH. is rolling in money anyways and has all of the latest
gadgets.

\subsection{Other submissions}

\section{Requested modules/funds}

\subsection{Basic Module}

For each applicant, we apply for funding within the Basic Module.

\subsubsection{Funding for Staff}\label{sec:positions}
\label{sec:positions:research}

We apply for the following positions. All run over the entire duration of the proposed project.

\paragraph*{Non-doctoral staff}\ednote{compute amount in elan and copy here}

One doctoral researcher for 2 years at $100 \%$ for Michael Kohlhase.

One doctoral researcher for 2 years at $100 \%$ for Florian Rabe.

%\paragraph*{Postdoctoral staff}
%\ednote{postdoctoral researcher and comparable}

\paragraph*{Other research assistants}\ednote{students with BSc.}

One student with BSc. for 2 years at $100 \%$ for Michael Kohlhase.

One student with BSc. for 2 years at $100 \%$ for Florian Rabe.

\subsubsection{Direct Project Costs}

\paragraph{7.1.2.1\quad Equipment up to \texteuro10,000, Software and Consumables}

None.  PC will cover the workspace, computing needs, and consumables for its staff as part
of the basic support.

\paragraph{7.1.2.2\quad Travel Expenses}
\label{sec:travel}

\begin{oldpart}{rework}
  The travel budget shall cover:
  \begin{itemize}
  \item visits to external collaborators. We expect two international visits. We estimate
    that each visit will be most effective, if the junior researchers can spend about 3
    weeks with the partners. Thus we estimate 2500 {\texteuro} per visit.
  \item visits to national conferences to disseminate the results of {\pn}. We expect
    one visit for each year for each of the three researchers. (3 x 3 x 1000 {\texteuro})
  \item visits to international conferences to disseminate the results of {\pn}. These
    are in particular the International Joint Conference on Document Engineering (DocEng)
    and the Tech User Group Meeting (TUG). We expect one visit for each proposed
    researcher and for each year. (3 x 3 x 1500 {\texteuro})
  \end{itemize}

  This sums up to a total amount of 32.500 {\texteuro} for travel expenses for the whole
  funding period of three years which is split into 16.250 {\texteuro} for each institute
  (PC and Jacobs University).
\end{oldpart}

\paragraph{7.1.2.3\quad Visiting Researchers} (excluding Mercator Fellows)
\label{sec:funds:visiting}

Total expenses \textbf{10.200 \texteuro}
\medskip

As explained in Section~\ref{sec:travel}, we expect 5 incoming research visits.  Assuming
an average duration of 3 weeks, we estimate the cost of one visit at 600 {\texteuro} for
traveling and 70 {\texteuro} per night for accommodation, amounting to 2040 \texteuro per
visit.

\paragraph{7.1.2.4\quad Expenses for Laboratory Animals}

\paragraph{7.1.2.5\quad Other Costs}

\paragraph{7.1.2.6\quad Project-related Publication Expenses}

\subsubsection{Instrumentation}

\paragraph{7.1.3.1\quad Equipment exceeding \texteuro10,000}

\paragraph{7.1.3.2\quad Major Instrumentation exceeding \texteuro50,000}

\subsection{Module Temporary Position for Principal Investigator}

\subsection{Module Replacement Funding}

\subsection{Module Temporary Clinician Substitute}

\subsection{Module Mercator Fellows}

\subsection{Module Workshop Funding}

\subsection{Module Public Relations Funding}

% \subsubsection{Expenses for Laboratory Animals} None.
% \subsubsection{Other Costs} None.
% \subsubsection{Project-Related Publication Expenses} None.
% \subsection{Funding for Instrumentation} None.

%%% Local Variables: 
%%% mode: LaTeX
%%% TeX-master: "proposal"
%%% End: 

% LocalWords:  ipower texteuro sec;funds:direct oldpart medskip
