\section{Starting Point}

\subsection{State of the art and preliminary work}\label{sec:state}
\begin{todo}{from the proposal guidelines}
  For new proposals please explain briefly and precisely the state of the art in your
  field in its direct relationship to your project. This description should make clear in
  which context you situate your own research and in what areas you intend to make a
  unique, innovative, promising contribution. This description must be concise and
  understandable without referring to additional literature.

  For renewal proposals, please report on your previous work. This report should also be
  understandable without referring to additional literature.

  To illustrate and enhance your presentation you may refer to your own and others’
  publications. Indicate whenever you are referring to other researchers’ work.  Please
  list all cited publications in your bibliography under section 3. This reference list is
  not considered your list of publications. Note that reviewers are not required to read
  any of the works you cite. This also applies to review sessions that are held on
  site. In this case, manuscripts and publications that provide more information on the
  progress reports and are published up to the review panel’s meeting may be made
  available at the meeting to enable reviewers to read through the information. Reviews
  will be based only on the text of the actual proposal.
\end{todo}

\subsection{Project-related publications}\label{sec:projpapers}

\begin{todo}{from the proposal template}
  Please list your most significant publications that relate directly to the proposed project and document your preliminary work. This list serves as an important basis for assessing your proposal.
  
  Please not the ``Guidelines for Publication Lists''. 
  
  \url{www.dfg.de/formulare/1_91}
  
  The DFG may reject any proposals not in compliance with the rules on publication lists.
  
  If you are submitting a proposal to the DFG for the first time and have therefore not published in the proposed project area, please list only the up to ten most important publications that are part of your curriculum vitae (see C. Appendices).
\end{todo}

\subsubsection{Articles published by outlets with scientific quality assurence, book publications, and works accepted for publication but not yet published}\label{sec:peer-rev}

\dfgprojpapers[articles,books,confpapers]{Kohlhase:pdpl10,Lamport:ladps94,Knuth:ttb84,KohDavGin:psewads11,Lange:OpenMathCDLinkedData10,providemore}
\ednote{Anmerkung Jens: Ein nützliches Feature wäre hier, wenn das Paket eine (eventuell
  über Optionen der Dokumentklasse unterdrückbare) Warnung ausgeben würde, wenn zu viele
  Publikationen entsprechend DFG-Richtlinien angegeben werden. Die Anzahl ist sehr eng
  begrenzt.}

\subsubsection{Other pulications, both peer-reviewed and non-peer-reviewed\qquad None.}
\subsubsection{Patents\qquad None.}
\paragraph{1.2.3.1\quad Pending}
\paragraph{1.2.3.2\quad Issued}
%%% Local Variables: 
%%% mode: LaTeX
%%% TeX-master: "proposal"
%%% End: 

% LocalWords:  subsubsections dfgprojpapers pdpl10 providemore compactdesc
% LocalWords:  ourpubs nociteprolist KohKoh ccbssmt09 KohRabZho tmlmrsca10
% LocalWords:  Hutter09 sifemp09
