% The main file for developing the proposal. 
% Variants with different class options are 
% - submit.tex (no draft stuff, no ednotes, no revision information)
% - public.tex (like submit.tex, but no financials either) 
\providecommand{\classoptions}{,keys} % to be overwritten in variants
\documentclass[    % the options control the appearance, see the documentation
    gitinfo,       % show GIT information 
    RAM,           % also manage research assistant (HiWi) months
    general,       % make a title page, etc. 
    \classoptions] % the other class options from above, they can be overwritten e.g. in submit/final,tex
    {dfgproposal}
\addbibresource{../lib/dummy.bib}
\renewcommand{\familydefault}{\sfdefault}
\usepackage[scaled=.90]{helvet}
\usepackage[utf8]{inputenc}

\begin{document}

\begin{center}\color{red}\huge
  This mock proposal is just an example for \texttt{dfgproposal.cls} it reflects the 
  current DFG template valid from October 2011.
\end{center}

% \urldef{\gcpubs}\url{http://www.pcg.phony/~gc/pubs.html}
% \urldef{\mikopubs}\url{http://kwarc.info/kohlhase/publications.html}
\begin{proposal}[
  PI=miko,
  mikoname=Michael Kohlhase,
  mikoaffiliation=FAU Erlangen N\"urnberg,
  mikodept=Computer Science,
  mikotitle=Prof. Dr.,
  PI=gc,
  gcname=Michael Kohlhase,
  gcaffiliation=Power Consulting GmbH,
  gcdept=Science Affairs,
  gctitle=Dr.,
  site=fau,
  fauacronym=FAU,
  faushortname=FAU Erlangen N\"urnberg,
  site=pcg,
  pcgacronym=PCG,
  pcgshortname=Power Consulting GmbH,
  thema=Intelligentes Schreiben von Antr\"agen,
  acronym={iPoWr},
  acrolong={\underline{I}ntelligent} {\underline{P}r\underline{o}posal} {\underline{Wr}iting},
  title=\pn: \protect\pnlong,
  totalduration=3 years,
  start=1. Feb. 2010,
  months=24,
%  pcgRM=36, pcgRAM=36, fauRM=36, fauRAM=36,
  discipline=Computer Science, 
  areas=Knowledge Management,
  keywords={LaTeX Active Documents}
  ]

% It is often good to separate the top-level sections into separate files. 
% Especially in collaborative proposals. We do this here. 
\input{zusammenfassung}
\input{summary}
\section{Starting Point}

\subsection{State of the art and preliminary work}\label{sec:state}
\begin{todo}{from the proposal guidelines}
  For new proposals please explain briefly and precisely the state of the art in your
  field in its direct relationship to your project. This description should make clear in
  which context you situate your own research and in what areas you intend to make a
  unique, innovative, promising contribution. This description must be concise and
  understandable without referring to additional literature.

  For renewal proposals, please report on your previous work. This report should also be
  understandable without referring to additional literature.

  To illustrate and enhance your presentation you may refer to your own and others’
  publications. Indicate whenever you are referring to other researchers’ work.  Please
  list all cited publications in your bibliography under section 3. This reference list is
  not considered your list of publications. Note that reviewers are not required to read
  any of the works you cite. This also applies to review sessions that are held on
  site. In this case, manuscripts and publications that provide more information on the
  progress reports and are published up to the review panel’s meeting may be made
  available at the meeting to enable reviewers to read through the information. Reviews
  will be based only on the text of the actual proposal.
\end{todo}

\subsection{Project-related publications}\label{sec:projpapers}

\begin{todo}{from the proposal template}
  Please list your most significant publications that relate directly to the proposed project and document your preliminary work. This list serves as an important basis for assessing your proposal.
  
  Please not the ``Guidelines for Publication Lists''. 
  
  \url{www.dfg.de/formulare/1_91}
  
  The DFG may reject any proposals not in compliance with the rules on publication lists.
  
  If you are submitting a proposal to the DFG for the first time and have therefore not published in the proposed project area, please list only the up to ten most important publications that are part of your curriculum vitae (see C. Appendices).
\end{todo}

\subsubsection{Articles published by outlets with scientific quality assurence, book publications, and works accepted for publication but not yet published}\label{sec:peer-rev}

\dfgprojpapers[articles,books,confpapers]{Kohlhase:pdpl10,Lamport:ladps94,Knuth:ttb84,KohDavGin:psewads11,Lange:OpenMathCDLinkedData10,providemore}
\ednote{Anmerkung Jens: Ein nützliches Feature wäre hier, wenn das Paket eine (eventuell
  über Optionen der Dokumentklasse unterdrückbare) Warnung ausgeben würde, wenn zu viele
  Publikationen entsprechend DFG-Richtlinien angegeben werden. Die Anzahl ist sehr eng
  begrenzt.}

\subsubsection{Other pulications, both peer-reviewed and non-peer-reviewed\qquad None.}
\subsubsection{Patents\qquad None.}
\paragraph{1.2.3.1\quad Pending}
\paragraph{1.2.3.2\quad Issued}
%%% Local Variables: 
%%% mode: LaTeX
%%% TeX-master: "proposal"
%%% End: 

% LocalWords:  subsubsections dfgprojpapers pdpl10 providemore compactdesc
% LocalWords:  ourpubs nociteprolist KohKoh ccbssmt09 KohRabZho tmlmrsca10
% LocalWords:  Hutter09 sifemp09

\section{Objectives and work programme}\label{sec:workplan}

\subsection{Anticipated total duration of the project}\label{sec:duration}

\begin{todo}{from the proposal template}
Please state
\begin{itemize}
 \item the project's intended duration\footnote{Please refer to DFG form 1.01 for information on long-term projects, \url{www.dfg.de/formulare/1_01}.} and how long DFG funds will be necessary,
 \item for ongoing projects: since when the project has been active.
\end{itemize}
\end{todo}

\subsection{Objectives}\label{sec:objectives}

\begin{todo}{from the proposal template}
	Please give a concise description of your project's research programme and scientific objectives.
	
	Please indicate if you anticipate results that may be relevant to fields other than science (such as science policy, technology, the economy or society).
\end{todo}

\begin{objective}[id=firstobj,title=Supporting Authors]
  This is the first objective, after all we have to write proposals all the time, and we
  would rather spend time on research. 
\end{objective}

\begin{objective}[id=secondobj,title=Supporting Reviewers]
  They are only human too, so let's have a heart for them as well. 
\end{objective}


\subsection{Work programme including proposed research methods}\label{sec:wawp}

%%%%%%%%%%%%%%%%%%%%%%%%%%%%%%%%%%%%%%%%%%%%%%%%%%%%%%%%%%%%%%%%%%%%%%%%%%%%%%%%%
\LaTeX is the best document markup language, it can even be used for literate
programming~\cite{DK:LP,Lamport:ladps94,Knuth:ttb84}


\begin{todo}{from the proposal template}
For each applicant

Please give a detailed account of the steps planned during the proposed funding period. (For experimental projects, a schedule detailing all planned experiments should
be provided.)

The quality of the work programme is critical to the success of a funding proposal. The tasks to be performed within the
work programme should correspond to the funds requested. The work programme should therefore indicate and justify what types of funding will be needed and how the funds will be used, providing details on the individual items requested where applicable.

Please provide a detailed description of the methods that you plan to use in the project:
What methods are already available? What methods need to be developed? What assistance is needed from outside your own group/institute?

Please list all \textbf{cited} publications pertaining to the description of your work programme
in your bibliography under section 3.
\end{todo}

The project is organized around \pdatacount{all}{wp} work packages, which we summarize in
Figure~\ref{fig:wplist}. 
 
\wpfig

We ensure the dissemination and creation of the periodic integrative reports containing
the periodic Project Management Report, the Project Management Handbook, an Knowledge
Dissemination Plan ({\WPref{management}}), the Proceedings of the Annual {\pn} Summer
School as well as non-public Dissemination and Exploitation plans ({\WPref{dissem}}), as
well as a report of the {\pn} project milestones.

\begin{workplan}   
\begin{workpackage}[id=management,title=Project Management,wphases=1-24!.3,
 RM=2,RAM=8]
  Based on the ``Bewilligungsbescheid'' of the DFG, and based on the financial and
  administrative data agreed, the project manager will carry out the overall project
  management, including administrative management.  A project quality handbook will be
  defined, and a {\pn} help-desk for answering questions about the format (first
  project-internal, and after month 12 public) will be established. The project management
  will consist of the following tasks
\begin{tasklist}
\begin{task}[id=foo,wphases=0-3]%,requires=\taskin{t1}{dissem}]

  To perform the administrative, scientific/technical, and financial management of the
  project 
\end{task}
\begin{task}[id=bar,wphases=13-17!.5]

  To co-ordinate the contacts with the DFG and other funding bodies, building on the
  results in \taskref{management}{foo}
\end{task}
\begin{task}
  To control quality and timing of project results and to resolve conflicts
\end{task}
\begin{task}
  To set up inter-project communication rules and mechanisms
\end{task}
\end{tasklist}

\end{workpackage}

\begin{workpackage}[id=dissem,title=Dissemination and Exploitation,
RM=8]

Much of the activity of a project involves small groups of nodes in joint work. This work
 package is set up to ensure their best wide-scale integration, communication, and
 synergetic presentation of the results. Clearly identified means of dissemination of
 work-in-progress as well as final results will serve the effectiveness of work within the
 project and steadily improve the visibility and usage of the emerging semantic services.


 The work package members set up events for dissemination of the research and
 work-in-progress results for researchers (workshops and summer schools), and for industry
 (trade fairs). An in-depth evaluation will be undertaken of the response of test-users.
 
 \begin{tasklist}
  \begin{task}[id=t1,wphases=6-7]
    sdfkj
  \end{task}
  \begin{task}[id=t2,wphases=12-13]
    sdflkjsdf
  \end{task}
  \begin{task}[id=t3,wphases=18-19]
    sdflkjsdf
  \end{task}
 \begin{task}[id=t4,wphases=22-24] 
 \end{task}
\end{tasklist}

Within two months of the start of the project, a project website will go live. This
website will have two areas: a members' area and a public area.\ldots
\end{workpackage}
 
\begin{workpackage}[id=class,
   title=A LaTeX class for EU Proposals,short=Class,
  RM=12,RAM=8]

We plan to develop a {\LaTeX} class for marking up EU Proposals

We will follow strict software design principles, first comes a
requirements analys, then \ldots
\begin{tasklist}
  \begin{task}[id=c1,wphases=0-2]
    sdfsdf
  \end{task}
  \begin{task}[id=c2,wphases=4-8]
    sdfsdf
  \end{task}
  \begin{task}[id=c3,wphases=10-14]
    sdfsdf
  \end{task}
  \begin{task}[id=c4,wphases=20-24]
    sdfsdfd
  \end{task}
\end{tasklist}
\end{workpackage} 

\begin{workpackage}[id=temple,title= Proposal Template,
  short=Template,RM=12]

We plan to develop a template file for {\pn} proposals

We abstract an example from existing proposals
\begin{tasklist}
  \begin{task}[id=temple1,wphases=6-12]
    sdfdsf 
  \end{task}
  \begin{task}[id=temple2,wphases=18-24]%,requires=\taskin{t3}{class}]
    sdfsdf
  \end{task} 
\end{tasklist}
\end{workpackage}

\begin{workpackage}[id=workphase,title=A work package without tasks,
  wphases=0-4!.5]
  
  And finally, a work package without tasks, so we can see the effect on the gantt chart
  in fig~\ref{fig:gantt}.
\end{workpackage}
\end{workplan} 

\ganttchart[draft,xscale=.45]

%%% Local Variables: 
%%% mode: LaTeX
%%% TeX-master: "proposal"
%%% End: 

% LocalWords:  workplan.tex wplist dfgcount wa mansubsus duratio ipower wpfig
% LocalWords:  ganttchart xscale workplan workarea pdataref dissem workpackage foo
% LocalWords:  tasklist taskin taskref sdfkj sdflkjsdf sdfsdf sdfsdfd sdfdsf pn
% LocalWords:  firstobj secondobj pdatacount WAref ednote OBJref wphases
% LocalWords:  ldots OBJtref workphase gantttaskchart



\section{Bibliography Concerning the State of the Art, the Research objectives, and the
  Work Programme}\label{sec:bib}

\begin{todo}{from the proposal template}
In this bibliography, list only the works you cite in your presentation of the state of the
art, the research objectives, and the work programme. This bibliography is not the list
of publications. Non-published works must be included with the proposal.
\end{todo}
\printbibliography[heading=empty,notcategory=featured]
% the following will not become part of the public proposal after all most of this is
% technical or confidential.
\ifpublic\else
\section{People/collaborations/funding}\label{sec:funds}

\subsection{Employment status information}

\subsection{First-time proposal data}

\subsection{Composition of the project group}

\subsection{Researchers in Germany with whom you have agreed to cooperate on this project}

\subsection{Researchers abroad with whom you have agreed to cooperate on this project}

\subsection{Researchers with whom you have collaborated scientifically within the past three years}

\subsection{Project-relevant cooperation with commercial enterprises}

\subsection{Project-relevant participation in commercial enterprises}

\subsection{Scientific equipment}

\subsection{Other submissions}

\section{Requested modules/funds}

\subsection{Basic Module}

For each applicant, we apply for funding within the Basic Module.

\subsubsection{Funding for Staff}\label{sec:positions}
\label{sec:positions:research}

We apply for the following positions. All run over the entire duration of the proposed project.

\paragraph*{Non-doctoral staff}\ednote{compute amount in elan and copy here}

One doctoral researcher for 2 years at $100 \%$ for Michael Kohlhase.

One doctoral researcher for 2 years at $100 \%$ for Florian Rabe.

%\paragraph*{Postdoctoral staff}
%\ednote{postdoctoral researcher and comparable}

\paragraph*{Other research assistants}\ednote{students with BSc.}

One student with BSc. for 2 years at $100 \%$ for Michael Kohlhase.

One student with BSc. for 2 years at $100 \%$ for Florian Rabe.

\subsubsection{Direct Project Costs}

\paragraph{7.1.2.1\quad Equipment up to \texteuro10,000, Software and Consumables}

None.  PC will cover the workspace, computing needs, and consumables for its staff as part
of the basic support.

\paragraph{7.1.2.2\quad Travel Expenses}
\label{sec:travel}

\begin{oldpart}{rework}
  The travel budget shall cover:
  \begin{itemize}
  \item visits to external collaborators. We expect two international visits. We estimate
    that each visit will be most effective, if the junior researchers can spend about 3
    weeks with the partners. Thus we estimate 2500 {\texteuro} per visit.
  \item visits to national conferences to disseminate the results of {\pn}. We expect
    one visit for each year for each of the three researchers. (3 x 3 x 1000 {\texteuro})
  \item visits to international conferences to disseminate the results of {\pn}. These
    are in particular the International Joint Conference on Document Engineering (DocEng)
    and the Tech User Group Meeting (TUG). We expect one visit for each proposed
    researcher and for each year. (3 x 3 x 1500 {\texteuro})
  \end{itemize}

  This sums up to a total amount of 32.500 {\texteuro} for travel expenses for the whole
  funding period of three years which is split into 16.250 {\texteuro} for each institute
  (PC and Jacobs University).
\end{oldpart}

\paragraph{7.1.2.3\quad Visiting Researchers} (excluding Mercator Fellows)
\label{sec:funds:visiting}

Total expenses \textbf{10.200 \texteuro}
\medskip

As explained in Section~\ref{sec:travel}, we expect 5 incoming research visits.  Assuming
an average duration of 3 weeks, we estimate the cost of one visit at 600 {\texteuro} for
traveling and 70 {\texteuro} per night for accommodation, amounting to 2040 \texteuro per
visit.

\paragraph{7.1.2.4\quad Expenses for Laboratory Animals}

\paragraph{7.1.2.5\quad Other Costs}

\paragraph{7.1.2.6\quad Project-related Publication Expenses}

\subsubsection{Instrumentation}

\paragraph{7.1.3.1\quad Equipment exceeding \texteuro10,000}

\paragraph{7.1.3.2\quad Major Instrumentation exceeding \texteuro50,000}

\subsection{Module Temporary Position for Principal Investigator}

\subsection{Module Replacement Funding}

\subsection{Module Temporary Clinician Substitute}

\subsection{Module Mercator Fellows}

\subsection{Module Workshop Funding}

\subsection{Module Public Relations Funding}

% \subsubsection{Expenses for Laboratory Animals} None.
% \subsubsection{Other Costs} None.
% \subsubsection{Project-Related Publication Expenses} None.
% \subsection{Funding for Instrumentation} None.

%%% Local Variables: 
%%% mode: LaTeX
%%% TeX-master: "proposal"
%%% End: 

% LocalWords:  ipower texteuro sec;funds:direct oldpart medskip

\section{Project Requirements \deu{(Voraussetzungen f\"ur die Durchf\"uhrung des Vorhabens)}}

\subsection{Employment status information \deu{(Angaben zur Dienststellung)}}

\begin{todo}{from the proposal template}
For each applicant, state the last name, first name, and employment status (including
duration of contract and funding body, if on a fixed-term contract).
\end{todo}

\subsection{First-time proposal data \deu{(Angaben zur Erstantragstellung)}}

\begin{todo}{from the proposal template}
Only if applicable: Last name, first name of first-time applicant.

If this is your first proposal, reviewers will consider this fact when assessing your pro-
posal. Previous proposals for research fellowships, publication funding, travel allow-
ances, or funding for scientific networks are not considered first proposals. If you are
submitting a “first-time proposal” and it is part of a joint proposal, please note that your
independent project must be distinct from the other projects.

If you have already submitted a proposal as an applicant for a research grant and have
received a letter informing you of the funding decision, or if you have led an independ-
ent junior research group or project in a Collaborative Research Centre or Research
Unit, you are no longer eligible to submit a “first proposal”. If you have submitted a
“first-time proposal” and it was rejected, you may resubmit the application, in revised
form, as a first-time proposal for the same project.
\end{todo}

\subsection{Composition of the project group \deu{(Zusammensetzung der Projektarbeitsgruppe)}}

\begin{todo}{from the proposal template}
List only those individuals who will work on the project but will not be paid out of the
project funds. State each person’s name, academic title, employment status, and type
of funding.

Please list separately the individuals paid by your institution and those paid using other
third-party funding (including fellowships).
\end{todo}
\begin{sitedescription}{fau}
The KWARC (Knowledge Adaptation and Reasoning for Content) research group headed by
Michael Kohlhase for has the following members
\begin{compactdesc}
\item[Dr. N.N.] is the \ldots She has a background in\ldots.
\end{compactdesc}
Additionally, the group has attracted about 10 undergraduate and master's students that
actively take part in the project work and various aspects of research.
\end{sitedescription}


\subsection{Cooperation with other researchers \deu{(Zusammenarbeit mit anderen Wissenschaftlerinnen und Wissenschaftlern)}}

\subsubsection{Researchers with whom you have agreed to cooperate on this project \deu{(Wissenschaftlerinnen und Wissenschaftler, mit denen für dieses Vorhaben eine konkrete Vereinbarung zur Zusammenarbeit besteht)}}

\begin{compactdesc}
\item[Prof. Dr. Super Akquisiteur (Uni Paderborn)] knows exactly what to do to get funding
  with DFG, we will interview him closely and integrate all his intuitions into the
  {\pn} templates.
\item[Prof. Dr. Habe Nichts (Uni Hinterpfuiteufel)] has never gotten a grant proposal
  through with DFG, we will try to avoid his mistakes.
\item[Dr. Sach Bearbeiter (DFG)] will consult with the DFG requirements to be met in the
  proposals.
\item[Dr. Donald Knuth (Stanford University)] is so surprised that we want to do grant
  proposals in {\TeX/\LaTeX} that he will help us with any problems we have in coding in
  this wonderful programming language.
\end{compactdesc}

\subsubsection{Researchers with whom you have collaborated scientifically within the past three years \deu{(Wissenschaftlerinnen und Wissenschaftler, mit denen in den letzten drei Jahren wissenschaftlich zusammengearbeitet wurde)}}

\ednote{Anmerkung Jens: Etwas unklar, was die DFG hier möchte. Die Liste der Personen kann
  sehr lang sein, also ist es wahrscheinlich besser nur die wichtigsten Projekte und
  Kontakte zu listen.}

\begin{todo}{from the proposal template}
This information will assist the DFG’s Head Office in avoiding potential conflicts of in-
terest during the review process.
\end{todo}


\subsection{Scientific equipment \deu{(Apparative Ausstattung)}}

Jacobs University provides laptops or desktop workstations for all academic
employees. Great Consulting GmbH. is rolling in money anyways and has all of the latest
gadgets.


\subsection{Project-relevant interests in commercial enterprises \deu{(Projektrelevante Beteiligungen an erwerbswirtschaftlichen Unternehmen)}}

Not applicable.

%%% Local Variables: 
%%% mode: LaTeX
%%% TeX-master: "proposal"
%%% End: 

% LocalWords:  Durchf uhrung subsubsection ipower Hinterpfuiteufel Sach Aktivit
% LocalWords:  Erkl arungen


\section{Additional Information}\label{sec:additional}

Funding proposal XYZ-83282 has been submitted prior to this proposal on related topic XYZ.
\fi % ifpublic
\end{proposal}
\end{document}
 
%%% Local Variables: 
%%% mode: LaTeX
%%% TeX-PDF-mode:t
%%% TeX-master: t
%%% End: 

% LocalWords:  empty bibflorian systems rabe institutions modal historical pub ednotes
% LocalWords:  kwarc till formalsafe miko gc ipower ipowerlong Antr agen Beitr gcpubs

% LocalWords:  acrolong intellegible kollaboratives koh arenten ussen Proze pcg miko,PI
% LocalWords:  Versionsmanagementsystem textsc unterst utzt konzentieren stex
% LocalWords:  mechanik workplan thispagestyle newpage Principcal cvpubsmiko pn
% LocalWords:  ourpubs zusammenfassung printbibliography pubspage ntelligent
% LocalWords:  iting pnlong dfgproposal.cls
