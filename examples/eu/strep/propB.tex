% The main file for developing the proposal. 
% Variants with different class options are 
% - submit.tex (no draft stuff, no ednotes, no revision information)
% - public.tex (like submit.tex, but no financials either) 
\providecommand{\classoptions}{,keys} % to be overwritten in variants
\documentclass[noworkareas,deliverables,gitinfo,report\classoptions]{euproposal}
\usepackage{nimbusserif}
\addbibresource{../lib/dummy.bib}

\begin{document}
\begin{center}\color{red}\huge
  This mock proposal is just an example for \texttt{euproposal.cls} it reflects the ICT
  template of January 2012
\end{center}
\begin{proposal}[
  % Site 1
  site=fau,
  fauurl=http://fau.de,
  faustreetaddress={Martenstr. 3},
  fautownzip={91058 Erlangen},
  faucountryshort=D,
  faucountry=Germany,
  fautype=University,
  faulogo=jacobs-logo.png,
  fauacronym=FAU,
  faushortname=FAU Erlangen N\"urnberg,
  fauRM=36,
  % Site 2
  site=efo,
  efoshortname=European Future Office,
  efourl=http://efo.eu,
  efocountryshort=NL,
  efocountry={The Netherlands}, 
  efotownzip={Utrecht, 3kd89},
  efostreetaddress={Kruislann 777},
  efotype=NGO,
  efoacronym=EFO,
  efoRM=36, 
  % Site 3
  site=bar,
  barurl=http://bar.fr,
  barcountryshort=F,
  barcountry={France},
  barstreetaddress={Rue de Montparnasse}
  bartownzip={Paris},
  bartype=University,
  baracronym=BAR,
  barshortname=Universit\'e de BAR,
  barRM=36,
  % Site 4
  site=baz,
  bazurl=http://baz.co.uk,
  bazcountryshort=UK,
  bazcountry=United Kingdom,
  bazstreetaddress={4711 Silicon Glen Drive},
  baztownzip={Westerfield U3F2B},
  baztype=SME,
  bazshortname=BAZ International,
  bazacronym=BAZ,
  bazRM=36,
  % Proposal metadata
  acronym={iPoWr},
  acrolong={\underline{I}ntelligent} {\underline{P}r\underline{o}sal} {\underline{Wr}iting},
  title=\pn: \protect\pnlong,
  callname = ICT Call 1,
  callid = FP7-???-200?-?,
  instrument= Small or Medium-Scale Focused Research Project (STREP), 
  challengeid = 4,
  challenge = ICT for EU Proposals,
  objectiveid={ICT-2012.4.4}, 
  objective = Technology-enhanced Documents,
  outcomeid = b1,
  outcome = {More time for Research, not Proposal writing},
  months=24, 
  % Coordinator
  coordinator=Prof. Dr. Michael Kohlhase,
  Cemail=michael.kohlhase@fau.de,
  Ctelfax=(49) 9131-85-64052/55, 
  compactht]
\begin{abstract}
  Writing grant proposals is a collaborative effort that requires the integration of
  contributions from many individuals. The use of an ASCII-based format like {\LaTeX}
  allows to coordinate the process via a source code control system like
  {\textsc{Subversion}}, allowing the proposal writing team to concentrate on the contents
  rather than the mechanics of wrangling with text fragments and revisions.
\end{abstract}

\tableofcontents

\begin{todo}{from the proposal template}
  Recommended length for the whole part B: 50--60 pages (including tables, references,
  etc.)
\end{todo}
\svnInfo $Id: quality.tex 22870 2012-01-02 11:01:58Z kohlhase $
\svnKeyword $HeadURL: https://svn.kwarc.info/repos/kwarc/doc/macros/forCTAN/proposal/eu/examples/fetopenstrep/quality.tex $
\chapter{Scientific and Technical Quality}\label{chap:quality}
\begin{todo}{from the proposal template}
  Maximum length for the whole of Section 1 –-- twenty pages, not including the tables in
  Section 1.3
\end{todo}

\svnInfo $Id: breakthrough.tex 21554 2011-04-30 05:53:27Z kohlhase $
\svnKeyword $HeadURL: https://svn.kwarc.info/repos/kwarc/doc/macros/forCTAN/proposal/eu/examples/fetopenstrep/breakthrough.tex $
\section{Targeted breakthrough and long-term vision}\label{sec:objectives}
\begin{todo}{from the proposal template}
  Describe the breakthrough(s) that you are targeting to achieve. What is the long-term
  vision (scientific, technological, societal, other) that motivates this breakthrough?
  Explain how this breakthrough is an essential step towards the achievement of your
  long-term vision, in particular in terms of new forms and uses of information and
  information technologies. Describe the concrete objectives that you consider to
  constitute the proof-of-concept of such a breakthrough. The objectives should be those
  that you consider achievable within the project, in spite of the inherent risks. They
  should be stated in a verifiable form, including through the milestones that will be
  indicated under Section 1.3 below.
\end{todo}

%%% Local Variables: 
%%% mode: latex
%%% TeX-master: "propB"
%%% End: 

\input{novelty}
\input{methodology}
%%% Local Variables: 
%%% mode: LaTeX
%%% TeX-master: "propB"
%%% End: 

\newpage
\include{implementation}\newpage
\include{impact}\newpage
\end{proposal}
\include{members}
\include{issues}
\end{document}

%%% Local Variables: 
%%% mode: LaTeX
%%% TeX-master: t
%%% End: 

% LocalWords:  efo efoRM baz bazRM miko acrolong ntelligent iting pn pnlong
% LocalWords:  textsc newpage compactht texttt euproposal.cls callname callid
% LocalWords:  challengeid objectiveid outcomeid tableofcontents
